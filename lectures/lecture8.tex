\section{Canonical (Anti)-Commutation Relations}
\subsection{Annihilation Operator, Notation, and Matrix Elements}
As in the end of last time, we are trying to discuss the operators on $\Gamma_s(H)$. The hard part is that these operators are going to be acting on an infinite-dimensional Hilbert space. To make this easier, we intoduce a preferred basis - one that we know and understand very very well. We will henceforth consider only linear combinations of the operators. This baiis is going to be the creation and annihilation operators.

In unsymmetrized Fock space (that is, $\bigoplus_{\NN} \Gamma(\Hh)$), the creation and annihilation operators are proportional to rather curious objects. We have $a^\dagger(\ket{e_j}) \propto o^*_j$, where $o_j^*$ are the generators of what is called the Cuntz Algebra $O_n$. This algebra has many remarkable properties, but sadly we will not be able to go into this in detail.

We looked at the creation operator in the last lecture, so the task now is to describe it's adjoint. You will notice that we gave the creation operator a dagger in anticipation of taking the adjoint. 

\begin{defn}[Annihilation Operator] We define the annihilation operator in terms of its action on a vector.
\begin{equation}
a_s(\ket{\phi})P_s(\ket{\psi_1}\otimes...\otimes\ket{\psi_N})
= \frac{1}{\sqrt{N}}\sum_{k=1}^N s^{k-1} P_s(\ket{\psi_1}\otimes...\otimes\ket{\psi_{k-1}}\otimes\braket{\phi}{\psi_k}\ket{\psi_{k+1}}\otimes...\otimes\ket{\psi_N})
\end{equation}
\end{defn}

It is an exercise to show that this is indeed the adjoint of the creation operator. Remind yourself that the adjoint is defined as the operator fulfilling $\braket{A\psi}{\phi}=\braket{\psi}{A^\dagger \phi}$ and use the fact that to extend the inner product to $\Gamma_s(\Hh)$ you need to sum over all the inner products of the components. This brings us to the algebra that was foreshadowed at the start of the lecture.

Let $M = \annih{\ket{\phi_1}}\create{\ket{\phi_2}}$, and $M'=\create{\ket{\phi_1}}\annih{\ket{\phi_2}}$. These operators preseve the total number of particles, so when acting on some state $P_s(\ket{\psi_1}\otimes...\otimes\ket{\psi_N})$ we find the same result, differing by a sign.

\[M=\braket{\phi_1}{\phi_2}\II+sM'\]

To summarize, we have the following commutator (s-bracket) relation:

\begin{equation}
[\create{\ket{\phi_1}},\annih{\ket{\phi_2}}]_s=\braket{\phi_1}{\phi_2}\II
\end{equation}

Where $[\cdot,\cdot]_+$ is the anticommutator, and $[\cdot,\cdot]_-$ is the commutator.

The algebras generated by these operators $a_s,a^\dagger_s$ are called the CAR (canonical anticommutation relation) algebra and the CCR (canonical commutation relation) algebra. There is a huge difference between the two algebras - the creation and annihilation operators are not bounded operators in the Boson case, so to generate the CCR algebra properly we need to take exponentials of the operators. 

Something to keep in mind is that we typically have a specific basis in mind when using the operators, so we could use the following notation.
\begin{equation}
a_\mu =: a_s(\ket{\phi_\mu})
\end{equation}
\begin{equation}
a_\mu^\dagger =: a_s^\dagger(\ket{\phi_\mu})
\end{equation}
We typically refer to the states $\phi_\mu$ as 'modes', or 'envelopes' due to the fact that much of the time we are describing waves. They are wavefunctions that specify a particle's state in a system, so it's natural to refer to them as modes when they are commonly sines, cosines, or sometimes Hermite polynomials. They almost always form a Fourier basis. 

Because the map which takes $\ket{\phi} \mapsto a_s(\ket{\phi})$ is antilinear, we can express any creation and annihilation operator as a linear combination of the basis creation and annihilation operators:

\begin{equation}
a_s(\ket{\phi}) = \sum_{\mu \in B}\phi_\mu^*a_\mu
\end{equation}

Where $\ket{\phi} = \sum \phi_\mu \ket{e_\mu}$. This notation gives us the following CCRs and CARs:

\[ [a_\mu,a_\nu]_s=[a_\mu^\dagger,a_\nu^\dagger]_s=0 \]
\[ [a_\mu,a_\nu^\dagger]_s=\delta_{\mu\nu}\II \]

In the occupation number basis we can work out the matrix elements of these operators. We start with the action on an occupation number state:

\[a_\mu^\dagger \ket{n} = s(\pi) \sqrt{n_\mu+1}\ket{n_1,...,n_\mu+1,n_{\mu+1},...}\]
\[a_\mu \ket{n} = s(\pi)\sqrt{n_\mu}\ket{n_1,...,n_\mu-1,...}\]

Here, $\pi$ is the permutation which sorts the sequence $\mu,\nu_1,\nu_2,...$ into ascending order (with the $\nu_i$'s being the remaing states you can occupy besides the $\mu$'th one). This is enough information to generate the matrix elements quite easily. This is important because if you work with Fermions you will want to make sure you sort everything because you will pick up signs. 
The following example can clarify the concept. 

\begin{example}
	Let $s = -1$, $H=\CC^3$, and $\Hh = H^{\otimes 3}$. Then we want to calculate $a_2^\dagger \ket{101}$. This should give some multiple of $\ket{111}$, but since $\ket{101}$ can be relabeled to give $-\ket{011}$  (which is in ascending order), we must introduce a negative sign. To clarify, this means that the permutation we are using is $\pi : (2,1,3)\mapsto (1,2,3)$, which is of odd parity. Therefore $a_2^\dagger \ket{101}=-\ket{111}$.
\end{example}

What makes this tricky to work out on a computer is the $-1$ factors. It's worth it to do these calculations for Fermions, and for Bosons you might already see that these will look very similar to the Harmonic Oscillator ladder operators. There's a faster way to do this, since at a large amount of particles the brute force method will be too slow. This is called the Jordan-Wigner Transform.
\subsection{Jordan-Wigner Transform}
To compute the matrix elements of the creation/annihilation operators with this method, we will need to use the Pauli matrices (which you have seen in your previous courses). These are:
\begin{equation}
\sigma^x = \begin{bmatrix}0&1\\1&0\end{bmatrix}, \sigma^y = \begin{bmatrix}0&-i\\i&0\end{bmatrix}, \sigma^z = \begin{bmatrix}1&0\\0&-1\end{bmatrix}, \sigma^0 = I
\end{equation}
We also need the projection matrices:
\begin{equation}
\sigma^+ = \begin{bmatrix}0&0\\1&0\end{bmatrix}, \sigma^- = \begin{bmatrix}0&1\\0&0\end{bmatrix}
\end{equation}

The Jordan-Wigner transformation is given by the following. It's probably easy to see that for the first matrix we have the following:

\[a_1^\dagger = \sigma_1^+\otimes \II_2\otimes...\otimes\II_n =: \sigma_0^\dagger \otimes \II_{2...n}\]
\[a_1 = \sigma_1^- \otimes \II_{2...n}\]

To get $a_2^\dagger$ we can't just insert a $\sigma_2^+$ into the tensor product, we need it to anticommute with $a_1^\dagger$. So we note that $\sigma_1^z$ anticommutes with $\sigma_1^+$. This gives:
\[a_2^\dagger = \sigma_1^z \otimes \sigma_2^+\otimes \II_{3...n}\]
\[a_2 = \sigma_1^z\otimes \sigma_2^- \otimes \II_{3...n}\]

In general the transform is given by this:
\begin{defn}[Jordan-Wigner Transform]
The Jordan-Wigner Transform generates the CAR algebra in terms of the Pauli matrices as follows:
\begin{equation}
a_j^\dagger = \sigma_1^z\otimes\sigma_2^z\otimes...\otimes \sigma_{j-1}^z \otimes \sigma_j^+ \otimes \II_{j+1...n}
\end{equation}
\begin{equation}
a_j = \sigma_1^z\otimes\sigma_2^z\otimes...\otimes \sigma_{j-1}^z \otimes \sigma_j^- \otimes \II_{j+1...n}
\end{equation}
\end{defn}

The $2^n$ by $2^n$ matrices given by the Jordan-Wigner transform completely generate the CAR. Here's something to ponder: what if we wanted to get away with smaller matrices? You can't if you want to be exact, but what if you only need to be approximately correct (ie $\{a_\mu,a\nu\} \approx \delta_{\mu \nu}\II$)?

One quick thing to define now is the number operator.
\begin{defn}[Number Operator] We define the number operator as follows:
\begin{equation}N_\mu =: a_\mu^\dagger a_\mu\end{equation}
\end{defn}
This has the effect that $N_\mu\ket{n} = n_\mu\ket{n}$. We will see that $\hat N = \sum N_\mu$, where $\hat N$ is the total particle number operator. This is easy to see, just apply the sum ot an arbitrary ket and factor the ket back out, leaving the sum of each of the $n_\mu$'s.
